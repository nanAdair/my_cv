\documentclass[10pt,a4paper]{moderncv}

\usepackage{fontspec,xunicode,xltxtra}

\defaultfontfeatures{Scale=MatchLowercase}
\setmainfont[Numbers=OldStyle,Mapping=tex-text]{Times New Roman}
\setsansfont[Mapping=tex-text]{Arial}
\setmonofont{Courier New}

\usepackage[BoldFont,SlantFont,CJKchecksingle,CJKnumber]{xeCJK}
%\setCJKmainfont[BoldFont={Adobe Heiti Std},ItalicFont={Adobe Kaiti Std}]{Adobe Song Std}
%\setCJKsansfont{Adobe Heiti Std}
%\setCJKmonofont{Adobe Fangsong Std}
%\setCJKmainfont[BoldFont={SimHei},ItalicFont={KaiTi}]{SimSun}
%\setCJKsansfont{SimHei}
%\setCJKmonofont{FangSong}
\setCJKmainfont[BoldFont={STHeiti},ItalicFont={STKaiti}]{STSong}
\setCJKsansfont{STHeiti}
\setCJKmonofont{STFangsong}
\punctstyle{hangmobanjiao}
\defaultfontfeatures{Mapping=tex-text}
\XeTeXlinebreaklocale "zh"
\XeTeXlinebreakskip = 0pt plus 1pt minus 0.1pt

\usepackage{xcolor}
\linespread{1.2}

\moderncvtheme[blue]{classic}

\usepackage[scale=0.95]{geometry}
\AtBeginDocument{\recomputelengths}

\setCJKfamilyfont{name}{STKaiti}
\newcommand\name{\CJKfamily{name}}

\firstname{\name{王秉楠}}
\familyname{}
%\title{个人简历}
\address{广州市番禺区大学城中山大学东校区至善园8号}{邮编:510006}
\mobile{18503726472}
\email{wangbn15@gmail.com}
%\homepage{wwinston.me}
\github{github.com/nanAdair}

%----------------------------------------------------------------------------------
%            content
%----------------------------------------------------------------------------------
\begin{document}
\maketitle

%\section{基本信息}
%\cvcomputer{姓\qquad 名:}{张三}{性\qquad 别:}{男}
%\cvcomputer{民\qquad 族:}{汉族}{出\qquad 生:}{xxxx年xx月xx日}

\section{实习经历}
\cvline{2015.5 -- 2015.7}{阿里云开放结构化数据服务团队(OTS)}
\cvline{}{$\bullet$ 设计与实现OTS Console(类似Hbase Shell的工具),提供给阿里内部用户,使其能更方便,快速地使用OTS服务。项目使用Java实现,上线一个月中用户使用情况良好。}
\cvline{}{$\bullet$ 设计与实现OTS Worker Request View,收集OTS存储模块在处理读请求时的数据流动,如查询的数据单元数,IO数等性能敏感数据,并实现命令供运维与开发人员实时监控。项目使用C++实现,已上线。}
\cvline{2015.7 -- 至今}{网易游戏平台服务部}
\cvline{}{$\bullet$ 参与Houston(内测应用分发平台)的开发,负责App后端接口开发,Web后端功能开发。项目使用Python实现。}

\section{项目经验}
%\cvline{年 -- 年}{简要描述}
%\cvline{}{$\bullet$ 详细描述子课题}
%\cvline{}{$\bullet$ 详细描述子课题}
\cvline{2014.10 -- 至今}{Tunnel目标代码混淆器}
\cvline{}{$\bullet$ Tunnel是在Scarab的框架中对目标代码进行混淆操作。它利用了ROP的攻击思想,在程序中插入栈内容构造的指令,通过栈指针来控制程序的执行,并改变原先指令的分布,从而将控制流隐藏。由于反汇编算法(逆向攻击的基础)只能按照固定的算法来进行反汇编操作,Tunnel的混淆使得反汇编算法错误地对二进制代码进行反汇编,从而破坏了逆向攻击。项目目前完成整体架构设计,仍在进行中。}
\cvline{2014.2 -- 2014.5}{Scarab目标代码混淆框架}
\cvline{}{$\bullet$ 为了代码混淆方法能够更加灵活,独立地进行混淆操作而提出的混淆框架。Scarab本质上是一个增强型链接器。它对输入的目标代码文件进行处理,获取该程序的段,符号,重定位等信息,并通过对目标文件进行反汇编操作创建程序的指令流以及控制流图,封装并定义混淆操作的API接口。在混淆操作完成后,Scarab继续重组二进制指令并进行后续的Patch操作,使得最终生成一个功能依旧的可执行程序。}
\cvline{}{$\bullet$ 职责:项目负责人,项目架构设计,链接器大部分代码实现}
%\cvline{}{$\bullet$ 核心技术:C++,ASM,valgrind,GDB}
\cvline{2013.2 -- 2013.5}{基于TPM的Linux可信启动的设计与实现}
\cvline{}{$\bullet$ 在Grub启动器中添加完整性度量流程,实现在当前阶段对下一段的启动文件进行度量和检测。并在度量失败时将系统引导到一个可信的内核环境中去,增强系统的可用性。}
\cvline{}{$\bullet$ 职责:个人独立开发,修改Grub并实现一个小型的启动加载器}
%\cvline{}{$\bullet$ 核心技术:C,可信计算}
%\cvline{2012.4 -- 2012.8}{基于完整性度量的进程监控器}
%\cvline{}{$\bullet$ Linux环境下的进程静态完整性度量的实现。利用内核中的IMA模块和Fnotify模块,在可执行文件加载到内存前首先对其完整性进行度量,使得被恶意篡改的程序无法再执行。}
%\cvline{}{$\bullet$ 职责:内核模块源码研究,理论学习,文档编写}
%\cvline{}{$\bullet$ 核心技术:C,Linux内核,可信计算}

\section{教育背景}
\cventry{2013年 -- 至今}{硕士}{中山大学}{计算机科学与技术专业}{广东省信息安全重点实验室}{}
\cventry{2009年 -- 2013年}{本科}{中山大学}{网络工程专业}{成绩在本专业排名前5\%(GPA:3.8/5)}{}

\section{专业技能}
\cvline{熟练的:}{C,C++,Python,ASM  Vim,Markdown,Git  Make,GDB  可信计算,逆向,密码学}
%\cvline{}{$\bullet$ 次要技能}
%\smallskip
\cvline{了解的:}{HTML,CSS,JS,Shell,Java  SublimeText,LaTex MySQL,Django  Web开发,Linux内核,机器学习}
%\cvline{}{$\bullet$ 等级考试}

\section{获奖情况}
\cvlistitem[$\bullet$]{2012年全国电子设计大赛信息安全邀请赛国家三等奖}
\cvlistitem[$\bullet$]{2010-2011学年国家励志奖学金}
\cvlistitem[$\bullet$]{大一,大二校二等奖学金,大三校三等奖学金}

%\section{个人爱好}
%\cvlistitem[$\bullet$]{主要爱好}
%\cvlistitem[$\bullet$]{次要爱好}

%\section{自我评价}
%\cvline{学习}{简要说明}
%\cvline{工作}{简要说明}
%\cvline{生活}{简要说明}

%\begin{thebibliography}{99}
    %\bibitem{1}Cai Qiwang, Cai guoyang, \textbf{Wang Bingnan} (2014). \textit{Construction of Trusted Chain of Secure Computing on Trusted USBKey}, WCEE 2015.
%\end{thebibliography}

\end{document}

